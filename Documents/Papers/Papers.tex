% Options for packages loaded elsewhere
\PassOptionsToPackage{unicode}{hyperref}
\PassOptionsToPackage{hyphens}{url}
\PassOptionsToPackage{dvipsnames,svgnames,x11names}{xcolor}
%
\documentclass[
  super,
  preprint,
  3p]{elsarticle}

\usepackage{amsmath,amssymb}
\usepackage{iftex}
\ifPDFTeX
  \usepackage[T1]{fontenc}
  \usepackage[utf8]{inputenc}
  \usepackage{textcomp} % provide euro and other symbols
\else % if luatex or xetex
  \usepackage{unicode-math}
  \defaultfontfeatures{Scale=MatchLowercase}
  \defaultfontfeatures[\rmfamily]{Ligatures=TeX,Scale=1}
\fi
\usepackage{lmodern}
\ifPDFTeX\else  
    % xetex/luatex font selection
\fi
% Use upquote if available, for straight quotes in verbatim environments
\IfFileExists{upquote.sty}{\usepackage{upquote}}{}
\IfFileExists{microtype.sty}{% use microtype if available
  \usepackage[]{microtype}
  \UseMicrotypeSet[protrusion]{basicmath} % disable protrusion for tt fonts
}{}
\makeatletter
\@ifundefined{KOMAClassName}{% if non-KOMA class
  \IfFileExists{parskip.sty}{%
    \usepackage{parskip}
  }{% else
    \setlength{\parindent}{0pt}
    \setlength{\parskip}{6pt plus 2pt minus 1pt}}
}{% if KOMA class
  \KOMAoptions{parskip=half}}
\makeatother
\usepackage{xcolor}
\setlength{\emergencystretch}{3em} % prevent overfull lines
\setcounter{secnumdepth}{5}
% Make \paragraph and \subparagraph free-standing
\ifx\paragraph\undefined\else
  \let\oldparagraph\paragraph
  \renewcommand{\paragraph}[1]{\oldparagraph{#1}\mbox{}}
\fi
\ifx\subparagraph\undefined\else
  \let\oldsubparagraph\subparagraph
  \renewcommand{\subparagraph}[1]{\oldsubparagraph{#1}\mbox{}}
\fi

\usepackage{color}
\usepackage{fancyvrb}
\newcommand{\VerbBar}{|}
\newcommand{\VERB}{\Verb[commandchars=\\\{\}]}
\DefineVerbatimEnvironment{Highlighting}{Verbatim}{commandchars=\\\{\}}
% Add ',fontsize=\small' for more characters per line
\usepackage{framed}
\definecolor{shadecolor}{RGB}{241,243,245}
\newenvironment{Shaded}{\begin{snugshade}}{\end{snugshade}}
\newcommand{\AlertTok}[1]{\textcolor[rgb]{0.68,0.00,0.00}{#1}}
\newcommand{\AnnotationTok}[1]{\textcolor[rgb]{0.37,0.37,0.37}{#1}}
\newcommand{\AttributeTok}[1]{\textcolor[rgb]{0.40,0.45,0.13}{#1}}
\newcommand{\BaseNTok}[1]{\textcolor[rgb]{0.68,0.00,0.00}{#1}}
\newcommand{\BuiltInTok}[1]{\textcolor[rgb]{0.00,0.23,0.31}{#1}}
\newcommand{\CharTok}[1]{\textcolor[rgb]{0.13,0.47,0.30}{#1}}
\newcommand{\CommentTok}[1]{\textcolor[rgb]{0.37,0.37,0.37}{#1}}
\newcommand{\CommentVarTok}[1]{\textcolor[rgb]{0.37,0.37,0.37}{\textit{#1}}}
\newcommand{\ConstantTok}[1]{\textcolor[rgb]{0.56,0.35,0.01}{#1}}
\newcommand{\ControlFlowTok}[1]{\textcolor[rgb]{0.00,0.23,0.31}{#1}}
\newcommand{\DataTypeTok}[1]{\textcolor[rgb]{0.68,0.00,0.00}{#1}}
\newcommand{\DecValTok}[1]{\textcolor[rgb]{0.68,0.00,0.00}{#1}}
\newcommand{\DocumentationTok}[1]{\textcolor[rgb]{0.37,0.37,0.37}{\textit{#1}}}
\newcommand{\ErrorTok}[1]{\textcolor[rgb]{0.68,0.00,0.00}{#1}}
\newcommand{\ExtensionTok}[1]{\textcolor[rgb]{0.00,0.23,0.31}{#1}}
\newcommand{\FloatTok}[1]{\textcolor[rgb]{0.68,0.00,0.00}{#1}}
\newcommand{\FunctionTok}[1]{\textcolor[rgb]{0.28,0.35,0.67}{#1}}
\newcommand{\ImportTok}[1]{\textcolor[rgb]{0.00,0.46,0.62}{#1}}
\newcommand{\InformationTok}[1]{\textcolor[rgb]{0.37,0.37,0.37}{#1}}
\newcommand{\KeywordTok}[1]{\textcolor[rgb]{0.00,0.23,0.31}{#1}}
\newcommand{\NormalTok}[1]{\textcolor[rgb]{0.00,0.23,0.31}{#1}}
\newcommand{\OperatorTok}[1]{\textcolor[rgb]{0.37,0.37,0.37}{#1}}
\newcommand{\OtherTok}[1]{\textcolor[rgb]{0.00,0.23,0.31}{#1}}
\newcommand{\PreprocessorTok}[1]{\textcolor[rgb]{0.68,0.00,0.00}{#1}}
\newcommand{\RegionMarkerTok}[1]{\textcolor[rgb]{0.00,0.23,0.31}{#1}}
\newcommand{\SpecialCharTok}[1]{\textcolor[rgb]{0.37,0.37,0.37}{#1}}
\newcommand{\SpecialStringTok}[1]{\textcolor[rgb]{0.13,0.47,0.30}{#1}}
\newcommand{\StringTok}[1]{\textcolor[rgb]{0.13,0.47,0.30}{#1}}
\newcommand{\VariableTok}[1]{\textcolor[rgb]{0.07,0.07,0.07}{#1}}
\newcommand{\VerbatimStringTok}[1]{\textcolor[rgb]{0.13,0.47,0.30}{#1}}
\newcommand{\WarningTok}[1]{\textcolor[rgb]{0.37,0.37,0.37}{\textit{#1}}}

\providecommand{\tightlist}{%
  \setlength{\itemsep}{0pt}\setlength{\parskip}{0pt}}\usepackage{longtable,booktabs,array}
\usepackage{calc} % for calculating minipage widths
% Correct order of tables after \paragraph or \subparagraph
\usepackage{etoolbox}
\makeatletter
\patchcmd\longtable{\par}{\if@noskipsec\mbox{}\fi\par}{}{}
\makeatother
% Allow footnotes in longtable head/foot
\IfFileExists{footnotehyper.sty}{\usepackage{footnotehyper}}{\usepackage{footnote}}
\makesavenoteenv{longtable}
\usepackage{graphicx}
\makeatletter
\def\maxwidth{\ifdim\Gin@nat@width>\linewidth\linewidth\else\Gin@nat@width\fi}
\def\maxheight{\ifdim\Gin@nat@height>\textheight\textheight\else\Gin@nat@height\fi}
\makeatother
% Scale images if necessary, so that they will not overflow the page
% margins by default, and it is still possible to overwrite the defaults
% using explicit options in \includegraphics[width, height, ...]{}
\setkeys{Gin}{width=\maxwidth,height=\maxheight,keepaspectratio}
% Set default figure placement to htbp
\makeatletter
\def\fps@figure{htbp}
\makeatother

\makeatletter
\makeatother
\makeatletter
\makeatother
\makeatletter
\@ifpackageloaded{caption}{}{\usepackage{caption}}
\AtBeginDocument{%
\ifdefined\contentsname
  \renewcommand*\contentsname{Table of contents}
\else
  \newcommand\contentsname{Table of contents}
\fi
\ifdefined\listfigurename
  \renewcommand*\listfigurename{List of Figures}
\else
  \newcommand\listfigurename{List of Figures}
\fi
\ifdefined\listtablename
  \renewcommand*\listtablename{List of Tables}
\else
  \newcommand\listtablename{List of Tables}
\fi
\ifdefined\figurename
  \renewcommand*\figurename{Figure}
\else
  \newcommand\figurename{Figure}
\fi
\ifdefined\tablename
  \renewcommand*\tablename{Table}
\else
  \newcommand\tablename{Table}
\fi
}
\@ifpackageloaded{float}{}{\usepackage{float}}
\floatstyle{ruled}
\@ifundefined{c@chapter}{\newfloat{codelisting}{h}{lop}}{\newfloat{codelisting}{h}{lop}[chapter]}
\floatname{codelisting}{Listing}
\newcommand*\listoflistings{\listof{codelisting}{List of Listings}}
\makeatother
\makeatletter
\@ifpackageloaded{caption}{}{\usepackage{caption}}
\@ifpackageloaded{subcaption}{}{\usepackage{subcaption}}
\makeatother
\makeatletter
\@ifpackageloaded{tcolorbox}{}{\usepackage[skins,breakable]{tcolorbox}}
\makeatother
\makeatletter
\@ifundefined{shadecolor}{\definecolor{shadecolor}{rgb}{.97, .97, .97}}
\makeatother
\makeatletter
\makeatother
\makeatletter
\makeatother
\journal{Political Analysis}
\ifLuaTeX
  \usepackage{selnolig}  % disable illegal ligatures
\fi
\usepackage[]{natbib}
\bibliographystyle{elsarticle-num}
\IfFileExists{bookmark.sty}{\usepackage{bookmark}}{\usepackage{hyperref}}
\IfFileExists{xurl.sty}{\usepackage{xurl}}{} % add URL line breaks if available
\urlstyle{same} % disable monospaced font for URLs
\hypersetup{
  pdftitle={Back to the Future bias},
  pdfauthor={Manoel Galdino; Davi Moreira; Carolina Dolleans},
  pdfkeywords={Time Series Cross Section, Collider, DAG},
  colorlinks=true,
  linkcolor={blue},
  filecolor={Maroon},
  citecolor={Blue},
  urlcolor={Blue},
  pdfcreator={LaTeX via pandoc}}

\setlength{\parindent}{6pt}
\begin{document}

\begin{frontmatter}
\title{Back to the Future bias \\\large{Collider bias in TSCS studies} }
\author[1]{Manoel Galdino%
\corref{cor1}%
\fnref{fn1}}
 \ead{mgaldino@usp.br} 
\author[2]{Davi Moreira%
%
\fnref{fn2}}
 \ead{davi.moreira@gmail.com} 
\author[3]{Carolina Dolleans%
%
\fnref{fn3}}
 \ead{cat@example.com} 

\affiliation[1]{organization={University of São Paulo, Political
Science},city={São Paulo},postcodesep={}}
\affiliation[2]{organization={Emory University, Political
Science},city={Atlatna},postcodesep={}}
\affiliation[3]{organization={Universidade Federal de
Pernambuco, Political Science},,postcodesep={}}

\cortext[cor1]{Corresponding author}
\fntext[fn1]{This is the first author footnote.}
\fntext[fn2]{We would like Fernando Limongi, Lorena Barberia, Umberto
Mignozetti and participants at the Political Science graduate seminar at
University of São Paulo and the Polmeth Latam session participants.}
\fntext[fn3]{Yet another author footnote.}
        
\begin{abstract}
Current applied Time Series Cross Section (TSCS) regressions analysis
aimed at identifying a causal effect is not aware that standard practice
of controls inclusion can introduce what we call back to the future
bias. In the present paper, we develop a tipoligy for current heuristics
for control inclusion and show that they can inadvertently introduce a
collider variable in the context of TSCS data that may bias the result
(back to the future bias). A review of a sample of applied TSCS papers
shows that this is a prevalent problem in the political science and
international relations literature. To address this matter, we present
an improved heuristic to build a DAG that properly controls for the
correct variables in the context of TSCS data analysis and avoid the
back to the future bias. Monte Carlo simulation shows that it works well
in finite samples.
\end{abstract}





\begin{keyword}
    Time Series Cross Section \sep Collider \sep 
    DAG
\end{keyword}
\end{frontmatter}
    \ifdefined\Shaded\renewenvironment{Shaded}{\begin{tcolorbox}[sharp corners, enhanced, interior hidden, borderline west={3pt}{0pt}{shadecolor}, frame hidden, boxrule=0pt, breakable]}{\end{tcolorbox}}\fi

\hypertarget{introduction}{%
\section{Introduction}\label{introduction}}

The analysis of Time Series Cross Section (TSCS) data plays a crucial
role in current research practices within political science,
particularly in the fields of International Relations and Comparative
Politics. TSCS data offer a powerful combination of spatial and temporal
features, making them valuable for empirical analysis.

\begin{figure}

{\centering \includegraphics{C:/Users/mczfe/Documents/Papers/Collider TSCS/beware_collider/images/pol change civil war dynamic.png}

}

\caption{DAg civil war dynamic}

\end{figure}

Let us return to our previous example of political change and civil war.
Suppose, like Hegre et al.~(2001), one wants to add other controls to a
regression, such as democracy level. The DAG now is as follows:

\begin{figure}

{\centering \includegraphics{C:/Users/mczfe/Documents/Papers/Collider TSCS/beware_collider/images/democracy_control.png}

}

\caption{DAg civil war with new control}

\end{figure}

Let's say that after a civil war onset, the democracy level of the
country will likely decrease. In our DAG, this means there is a missing
arrow from current civil war nodes to current democracy levels nodes.
This is know as potential reverse causality, which is dealt with by
lagging the control. Thus. we need to add this arrow and the DAG is now
described by the following graph:

\begin{figure}

{\centering \includegraphics{C:/Users/mczfe/Documents/Papers/Collider TSCS/beware_collider/images/emocracy_control_missing_arrow.png}

}

\caption{DAg civil war wit inclusion of missing arrowl}

\end{figure}

\hypertarget{step-2-of-back-to-the-future-bias-treatment-as-cause-of-control-z}{%
\subsection{step 2 of back to the future bias -- treatment as cause of
control
z}\label{step-2-of-back-to-the-future-bias-treatment-as-cause-of-control-z}}

In step 2, the researcher should look at the literature on the causes of
democracy levels. In standard research practice, this never happens,
because according to both control check and confounding checking
heuristics, there is no need to review the literature for potential
causes of the controls included in the regression and include them also
in the regression. We call it foreign literature review, because is a
review of literature not directed related to the research question at
hand, but not another (foreign) research question (that causes a
control?). In any case, let's say she does that and she finds that the
treatment of current research, politcal change, is a cause of democracy
levels, mediated by another variable, not observed in the current study,
\(U\). The new DAG reflecting this potential causal channel is as below:

\begin{figure}

{\centering \includegraphics{C:/Users/mczfe/Documents/Papers/Collider TSCS/beware_collider/images/foreign collider bias civil war.png}

}

\caption{back to the future bias - civil war DAG}

\end{figure}

\begin{figure}

{\centering \includegraphics{C:/Users/mczfe/Documents/Papers/Collider TSCS/beware_collider/images/basic_back_future_bias.png}

}

\caption{back to the future bias - one period DAG}

\end{figure}

\hypertarget{tables-coming-from-r}{%
\section{Tables coming from R}\label{tables-coming-from-r}}

Tables can also be generated using R chunks, as shown in
Table~\ref{tbl-simple} example.

\begin{Shaded}
\begin{Highlighting}[]
\NormalTok{knitr}\SpecialCharTok{::}\FunctionTok{kable}\NormalTok{(}\FunctionTok{head}\NormalTok{(mtcars)[,}\DecValTok{1}\SpecialCharTok{:}\DecValTok{4}\NormalTok{])}
\end{Highlighting}
\end{Shaded}

\hypertarget{tbl-simple}{}
\begin{longtable}[]{@{}lrrrr@{}}
\caption{\label{tbl-simple}Caption centered above table}\tabularnewline
\toprule\noalign{}
& mpg & cyl & disp & hp \\
\midrule\noalign{}
\endfirsthead
\toprule\noalign{}
& mpg & cyl & disp & hp \\
\midrule\noalign{}
\endhead
\bottomrule\noalign{}
\endlastfoot
Mazda RX4 & 21.0 & 6 & 160 & 110 \\
Mazda RX4 Wag & 21.0 & 6 & 160 & 110 \\
Datsun 710 & 22.8 & 4 & 108 & 93 \\
Hornet 4 Drive & 21.4 & 6 & 258 & 110 \\
Hornet Sportabout & 18.7 & 8 & 360 & 175 \\
Valiant & 18.1 & 6 & 225 & 105 \\
\end{longtable}

\hypertarget{equations}{%
\section{Equations}\label{equations}}

Here is an equation: \[ 
  f_{X}(x) = \left(\frac{\alpha}{\beta}\right)
  \left(\frac{x}{\beta}\right)^{\alpha-1}
  e^{-\left(\frac{x}{\beta}\right)^{\alpha}}; 
  \alpha,\beta,x > 0 .
\]

Inline equations work as well: \(\sum_{i = 2}^\infty\{\alpha_i^\beta\}\)

Please make sure that your manuscript follows the guidelines in the
Guide for Authors of the relevant journal. It is not necessary to
typeset your manuscript in exactly the same way as an article, unless
you are submitting to a camera-ready copy (CRC) journal.

For detailed instructions regarding the elsevier article class, see
\url{https://www.elsevier.com/authors/policies-and-guidelines/latex-instructions}

\hypertarget{bibliography-styles}{%
\section{Bibliography styles}\label{bibliography-styles}}

Here are two sample references: \citet{Feynman1963118}
\citet{Dirac1953888}.

By default, natbib will be used with the \texttt{authoryear} style, set
in \texttt{classoption} variable in YAML. You can sets extra options
with \texttt{natbiboptions} variable in YAML header. Example

\begin{verbatim}
natbiboptions: longnamesfirst,angle,semicolon
\end{verbatim}

There are various more specific bibliography styles available at
\url{https://support.stmdocs.in/wiki/index.php?title=Model-wise_bibliographic_style_files}.
To use one of these, add it in the header using, for example,
\texttt{biblio-style:\ model1-num-names}.

\hypertarget{using-csl}{%
\subsection{Using CSL}\label{using-csl}}

If \texttt{cite-method} is set to \texttt{citeproc} in
\texttt{elsevier\_article()}, then pandoc is used for citations instead
of \texttt{natbib}. In this case, the \texttt{csl} option is used to
format the references. By default, this template will provide an
appropriate style, but alternative \texttt{csl} files are available from
\url{https://www.zotero.org/styles?q=elsevier}. These can be downloaded
and stored locally, or the url can be used as in the example header.


\renewcommand\refname{References}
  \bibliography{bibliography.bib}


\end{document}
